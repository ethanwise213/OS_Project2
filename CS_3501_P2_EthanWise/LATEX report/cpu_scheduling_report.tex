\documentclass[12pt]{article}
\usepackage[utf8]{inputenc}
\usepackage{geometry}
\geometry{a4paper, margin=1in}
\usepackage{amsmath}

\title{CPU Scheduling Simulation Report}
\author{Ethan Wise}
\date{April 28, 2025}

\begin{document}

\maketitle

\section*{Introduction}
This report summarizes the design, implementation, and analysis of a CPU scheduling simulation project. The project implements two scheduling algorithms: Shortest Remaining Time First (SRTF) and Highest Response Ratio Next (HRRN). These algorithms are evaluated based on metrics such as waiting time, turnaround time, CPU utilization, and throughput.

\section*{Design}
The project is structured into three main components:
\begin{itemize}
    \item \textbf{Process Class:} Represents processes with attributes like arrival time, burst time, and computed metrics.
    \item \textbf{Scheduling Algorithms:} Implements SRTF and HRRN algorithms.
    \item \textbf{Test Cases:} Validates the algorithms using predefined scenarios.
\end{itemize}

\section*{Implementation}
The project is implemented in C\#, adhering to object-oriented principles. Key features include:
\begin{itemize}
    \item \textbf{SRTF Algorithm:} Selects the process with the shortest remaining burst time.
    \item \textbf{HRRN Algorithm:} Selects the process with the highest response ratio, calculated as:
    \[ \text{Response Ratio} = \frac{\text{Waiting Time} + \text{Burst Time}}{\text{Burst Time}} \]
    \item \textbf{Metrics:} Computes average waiting time, turnaround time, CPU utilization, and throughput.
\end{itemize}

\section*{Analysis}
The algorithms are evaluated using three test cases:
\begin{itemize}
    \item \textbf{SRTF:} Minimizes waiting time but may cause starvation for longer processes.
    \item \textbf{HRRN:} Balances fairness and efficiency by considering response ratios.
\end{itemize}
Results demonstrate trade-offs between efficiency and fairness, highlighting the suitability of each algorithm for different scenarios.

\section*{Conclusion}
This project successfully implements and evaluates two CPU scheduling algorithms. The analysis provides insights into their strengths and weaknesses, offering guidance for their application in real-world systems.

\end{document}